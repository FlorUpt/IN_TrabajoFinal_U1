\documentclass[10pt,a4paper]{article}
\usepackage[utf8]{inputenc}
\usepackage[spanish]{babel}
\usepackage{amsmath}
\usepackage{amsfonts}
\usepackage{amssymb}
\usepackage{graphicx}
\usepackage{natbib}
\usepackage{lineno}
\usepackage{ragged2e}
\usepackage{multicol}
\setlength\columnsep{37pt}
\usepackage{enumerate} 
\usepackage[left=1.94cm,top=1.59cm,right=1.9cm,bottom=0.59cm]{geometry} 
\usepackage{fancyhdr}
\usepackage{url}
\usepackage{cite} 


\begin{document}
		
		\begin{center}
			\huge \textbf{Business Model Canvas aplicado a una empresa} 
		\end{center}
		\vspace{\baselineskip}
		\begin{center}
			\includegraphics[scale=0.37]{./Imagenes/logo}
		\end{center}
		\begin{multicols}{2}
			\small
			\begin{center}
				Nelia Escalante Marón\\
				2014049551\\
				UPT - Ingenierí­a de Sistemas\\
				EPIS\\
				Tacna, Perú\\
				\vspace{\baselineskip}
				Yerson Coaquira Calizaya\\
				2015053225\\
				UPT - Ingenierí­a de Sistemas\\  
				EPIS\\
				Tacna, Perú\\                 
				\vspace{\baselineskip}
				Flor Condori Gutierrez\\
				2015053227\\
				UPT - Ingenierí­a de Sistemas\\ 
				EPIS\\	
				Tacna, Perú\\                 
				\columnbreak
				
				\vspace{\baselineskip}
				Christian Cespedes Medina\\
				2010036256\\
				UUPT - Ingenierí­a de Sistemas\\  
				EPIS\\	
				Tacna, Perú\\                

				\vspace{\baselineskip}
				Javier Octavio Arteaga Ramos \\
				2007028981\\
				UPT - Ingenierí­a de Sistemas\\  
				EPIS\\	
				Tacna, Perú\\                 

			\end{center}
			\normalsize			
		\end{multicols}
		\vspace{\baselineskip}
		\begin{multicols}{2}
		\textbf{\textit{\large Abstract}}\rule[1.5mm]{5mm}{0.1mm} El modelo Canvas cuenta con 9 bloques los cuales hacen referencia a las características de la empresa que se quiere crear. Debemos tener en cuenta que al inicio puede costarnos un poco insertar los datos necesarios en cada bloque, y eso puede deberse a que el modelo de negocio aun no está bien definido.
		
		Los 9 bloques que va a llenar segun este modelo son: Propuesta de Valor, Los canales, Relaciones con los Clientes, Fuentes de ingresos, Recurso clave, Activida des principales, Alianza Clave y Estructura de Precio.
				
		\textit{The Canvas model has 9 blocks which refer to the characteristics of the company you want to create. We must take into account that at the beginning it can cost us a bit to insert the necessary data in each block, and that may be because the business model is not yet well defined.}
		
		\textit{In this final work we will apply the blocks of the BMC model to a company, these are: value proposition, channels, customer relations, revenue sources, key resource, core activities, key alliance and price structure}
		
		\vspace{\baselineskip}
			
		\textbf{\textit{\large Keybwords}}\rule[1.5mm]{5mm}{0.1mm} BMC, Web 2.0, aplicacion del Modelo Canvas a una empresa, cloud
		
		\columnbreak
		
		\section{Marco Teórico} 
		
		Todo emprendedor posee una idea de negocio, sin embargo, su puesta en práctica no resulta fácil, y mucho menos sacarle rentabilidad. Por ello contamos con varias estrategias, entre ellas el modelo Canvas. Estas estrategias buscan asegurarnos de que nuestras iniciativas llegaran a tener a éxito. Sin embargo, no todos los modelos de negocio nos dan soluciones perfectas. Es por ello por lo que surge el modelo Canvas.\\
		
		Modelo que se ha convertido desde el 2008 como herramienta estrella en la gestión estratégica y empresarial de un negocio. El modelo Canvas permite ver y moldear en un solo folio, estructurado en nueve elementos, cual es el modelo de nuestro negocio.  Y lo mejor de todo, es tan sencillo que puede ser aplicado en cualquier escenario, ya sea una pequeña, mediana y gran empresa.  Además, no sólo sirve para las nuevas empresas sino también para aquellas que ya están establecidas. \\
		
		En definitiva, este modelo trata de aprender muy rápido sobre el mercado, en un corto tiempo y con el mínimo coste.  Con el objetivo de lograr un modelo que busque la agilidad y la reducción del tiempo en el desarrollo de iniciativas empresariales, para finalmente generar productos y servicios que cumplan con las necesidades de los clientes y aporten valor.\\
		
		\textbf{¿En qué consiste el modelo Canvas?}\\
		
		En la descripción de un modelo de negocio dividido en nueve módulos básicos que reflejen el método para obtener ingresos en una empresa.  
		
		\begin{enumerate}[1.]
			\item \textit{Segmentos de clientes}
			
			La empresa debe definir en el bloque de segmentos de clientes cuál es el nicho de mercado que pretende alcanzar. Con el fin de satisfacer mejor a los clientes, una empresa puede agruparlos en distintos segmentos con necesidades, comportamientos u otros atributos comunes. Un modelo de negocio puede definir uno o varios segmentos a los cuales va a servir (al mismo tiempo está descartando otros nichos de clientes).\\
						
			Sin el nicho o segmento de clientes no es rentable, la empresa no puede sobrevivir por mucho tiempo. Por ello el segmento se define si las necesidades requieren y justifican una producto o servicio distinto, si es posible llegar a ellos a través de canales de distribución diferentes, están dispuestos a pagar el precio por el valor que obtienen y si se obtienen beneficios sustanciales.\\
			
			El modelo Canvas establece que hay varios tipos de segmentos de clientes, tales como mercado masivo (donde el modelo de negocio no se distingue entre uno y otro, se centra en un grupo grande de clientes con necesidades más o menos similares), nicho de mercado (se centra en un segmento especifico de clientes especializados y sus necesidades particulares), segmentado (clientes del mismo nicho con necesidades ligeramente diferentes), diversificado (cuenta con varios nichos o segmentos de clientes).
			
			\item \textit{Propuestas de Valor}
			
			En este bloque se describe el paquete de productos y servicios que crean valor para un segmento especifico de clientes.\\
			
			El modelo define que una propuesta de valor es la razón por la que los clientes recurren a una empresa en lugar de otra para resolver su problema o necesidad. Es el conjunto o paquete de beneficios que una empresa ofrece a sus clientes. Implica responder a preguntas como: ¿Qué valor damos a los clientes?, ¿Cuál de los problemas de nuestros clientes estamos ayudando a resolver?, ¿Qué necesidades de los clientes estamos satisfaciendo?, ¿Cuáles paquetes de productos y servicios ofrecemos para cada segmento de clientes?\\
			
			Las propuestas de valor pueden ser:
			
			\begin{itemize}
				\item Innovadoras: Mejoran el producto, el proceso o sus beneficios en función de las necesidades del cliente.
				\item Funcionales: Pueden ser similares a otras existentes en el mercado, pero con más funciones y atributos.
				\item Novedosas: Al satisfacer un conjunto totalmente nuevo de necesidades que antes no eran percibidas.
				\item De alto rendimiento: Mejora el rendimiento del producto o servicio.
				\item Personalizada: Adaptación de productos y servicios a necesidades específicas de sus clientes.
				\item De diseño: Un producto puede destacar por un diseño superior.
				\item De marca: Los clientes encuentran valor en el simple hecho de utilizar y visualizar una marca específica.
				\item Precio: Reciben un valor similar a un precio inferior y ayuda a los clientes a reducir sus costos
				\item Accesible: Lograr que los clientes tengan acceso a productos y servicios que anteriormente no podían disponer
				\item Conveniencia y usabilidad: Hacer que los productos sean más fáciles de usar y que resuelvan una necesidad oportuna.
			\end{itemize}
			
			\item \textit{Canales}	
			
			En este bloque la empresa establece cómo se va a llevar los productos o servicios hasta sus clientes indicando los mecanismos de distribución, contacto, venta, soporte y mantenimiento.\\
						
			Los canales -dicen los creadores del modelo de negocios Canvas- cumplen varias funciones: sensibilizan a los clientes, ayudan a los clientes a evaluar la propuesta de valor de la empresa, permite que los clientes compren productos y servicios específicos, entregan la propuesta de valor a los clientes, y proporcionan atención posterior a la compra.\\
			
			Aquí, se debe responder a preguntas: ¿Cómo quiere llegar hasta los clientes?, ¿Cómo son nuestros canales integrados?, ¿Cuáles funcionan mejor?, ¿Cuáles son los más costosos y eficientes?, ¿Cómo las vamos a integrar a las rutinas del cliente?\\
			
			El modelo Canvas establece varias clases de canales: propios, directos, indirectos y asociados. Una empresa debe encontrar la combinación adecuada para satisfacer a los clientes. Para definir esta mezcla debe calcular cuáles son los costos de cada uno y cuáles le aportan más utilidades.
			
			\item \textit{Relaciones con el Cliente}
			
			Una empresa debe tener claro el tipo de relación que quiere establecer con cada segmento de clientes: adquisición de clientes, retención de clientes o impulsar las ventas (upselling).\\
			
			Al inicio se pueden establecer estrategias agresivas de adquisición de clientes; luego se pasa a estrategias para retener la clientela actual; y después para lograr que los clientes compren más.\\
			
			
		\end{enumerate}
		
		
			
			
			
		\end{multicols}	
			
\end{document}}